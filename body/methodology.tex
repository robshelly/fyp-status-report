\section{Methodology}
	\subsection{Agile}
	The design methodology chosen for this project is Agile. Agile takes an iterative approach toe designing and delivery products. It is a goal driven methodology that aims to build and deliver software in a iteratively and incrementally from the beginning of the project. This is in contrast with more traditional approaches such as Waterfall which deliver in one final stage. A notable aspect of Agile is user stories. The project is broken down is small sections of functionality which can be independently developed and delivers upon completion \citep{rasmusson}. 
	
	A particular Agile framework which will be used for this project is Scrum. SCrum organises development in to cycles of \textit{sprints}. A sprint consists short time-limited periods of development each with it's own development goals based on work within  the backlog. Each sprint will consists of regular update and a final review/retrospective before beginning the next sprint. The Scrum master keeps the sprint focused on its development goal \citep{scrum}.
	
	Scrum is an ideal model for developing this project. The project supervisor plays a role in line with the concept of a Scrum master. Also, the use of user stories means sprints can be aligned with the implementation of stories.
	
	\subsection{CI/CD with Jenkins}
	Continuous Integration/Continuous Deployment (or continuous Delivery) is a development concept that focuses on the frequent and automated testing building and releasing of code. It aims to remove the large workload required when it is time tp release a version or update of a product by performing the same process in a automated manner on every code commit \citep{pittet}.
	
	Continuous integration refers to preparing the code for release and often as code commits are performed. For example, running tests and building Docker images on each commit. IT means that code is prepared for release at each stage of development, instead of when it come to release time. This may occur often as many times a day \citep{ramos}.
	Continuous Deployment is a step beyond Continuous Integration. After code is prepared for release, the built code is deployed to a server. However, this may be a development server. Pushing the built code to production required a manual trigger. Continuous Delivery automates this final manual trigger, meaning the entire process of moving code through testing, building and deployment to production is entirely automated \citep{ellingwood}.
	
	For this project, CI/CD (continuous development in this case) will be implemented using a Jenins automated build server. The fronted web app will be built aa s Docker image and deployed to ECS by Jenkins on every code commit.
	
	\subsection{JIRA}
	// TODO
	\subsection{Tool n}
	\subsection{Testing Approach}
	//TODO